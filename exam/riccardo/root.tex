\documentclass{beamer}
\usetheme{Berlin}
\usecolortheme{beaver}
\usepackage{graphicx}
\usepackage[export]{adjustbox}
\usepackage{listings}
\usepackage{courier}
\usepackage{amsmath}
\usepackage{lmodern}% http://ctan.org/pkg/lm
\usepackage{mathtools}
\usepackage[font={scriptsize,it}]{caption}
\usefonttheme{professionalfonts}

% code listings format
\lstset{basicstyle=\footnotesize\ttfamily,breaklines=true}
\lstset{
  caption=\lstname
}

% inline code
\newcommand{\code}[1]{\texttt{#1}}

% command for centering boxes
\makeatletter
\newcommand*{\centerfloat}{%
  \parindent \z@
  \leftskip \z@ \@plus 1fil \@minus \textwidth
  \rightskip\leftskip
  \parfillskip \z@skip}
\makeatother

% command for adding figures
\newcommand{\fig}[3]{
  \begin{figure}[H]
  \centerfloat
    \includegraphics[width=\textwidth,height=#1,keepaspectratio]{figures/#2}
	\caption{#3}
  \end{figure}
}

% command for adding figures
\newcommand{\figNoCapt}[2]{
  \begin{figure}[H]
  \centerfloat
    \includegraphics[width=\textwidth,height=#1,keepaspectratio]{figures/#2}
  \end{figure}
}

% document info
\title{Exam presentation}
\subtitle{Assignment 2.1 and 3}
\author[Riccardo]{Riccardo~Miccini\inst{1}}
\institute[DTU]
{
	\inst{1}
	Technical University of Denmark\\
	Digital Communication
}
\date{\today}
\subject{Digital Communication}


% front page
\begin{document}
\frame{\titlepage}

% overview
\begin{frame}
	\frametitle{Overview}
	\begin{itemize}
		\item Assignment 2
		\begin{itemize}
			\item Eye diagrams
			\item Q function
			\item Matched filter
		\end{itemize}
		\item Assignment 3
		\begin{itemize}
			\item Link budget model
			\item $SNR$ and $P_E$
			\item Alternative modulations
		\end{itemize}
	\end{itemize}
\end{frame}

% part 2.1
\begin{frame}
	\frametitle{Eye diagram}
	\begin{itemize}
		\item Plot composed by overlaying segments of different bit sequences
		\item Can be generated with an oscilloscope
		\item Shows effects of \emph{inter-symbol interference}
		\item Provides a qualitative measure of the system performance
	\end{itemize}
\end{frame}

\begin{frame}
	\frametitle{Eye diagram characteristics}
	\fig{9cm}{eye1.png}{Eye diagram of baseband antipodal signal}
	\begin{description}
		\item[$A$] Difference between high and low levels
		\item[$A_j$] Difference between $A$ and the eye opening
		\item[$T_j$] Deviations from ideal timing
		\item[$T_b$] Bit time period
	\end{description}
\end{frame}

\begin{frame}
	\frametitle{Eye diagram at different bandwidths}
	\begin{columns}
		\column{0.55\linewidth}
			\fig{8cm}{eye2.png}{Eye diagram for normalized bandwidths 0.3, 0.7, 1.2}
		\column{0.45\linewidth}
			\begin{itemize}
				\item Low BW: high amplitude and timing jitter
				\item High BW: no ISI, chances of higher noise
			\end{itemize}
	\end{columns}
\end{frame}

% part 2.2
\begin{frame}
	\frametitle{Q function}
	\begin{itemize}
		\item
	\end{itemize}
\end{frame}

% part 2.3
\begin{frame}
	\frametitle{Matched filters}
	\begin{itemize}
		\item 
	\end{itemize}
\end{frame}

% part 3
\begin{frame}
	\frametitle{Link budget model}
	\begin{itemize}
		\item A way of estimating the power of a received signal
		\item Takes into account all the gains and losses of transmitter, channel, and receiver
		\begin{equation}
			P_R = \left(\frac{\lambda}{4 \pi d}\right)^2 \frac{P_T G_T G_R}{L_0};
		\end{equation}
		\item In decibels:
		\begin{equation}
			P_{R, dB} = 20\log_{10}\left(\frac{4 \pi d}{\lambda}\right) + {ERP}_{dB} + G_{R, dB} - L_{0, dB};
		\end{equation}
	\end{itemize}
\end{frame}

\begin{frame}
	\frametitle{Link budget model variables}
	\begin{description}
		\item[$(\frac{4 \pi d}{\lambda})^2$] Free-space loss
		\item[$ERP = P_T G_T$] Effective radiated power
		\item[$G_R$] Gain of receiver antenna
		\item[$L_0$] Other losses, link budget margin
	\end{description}
\end{frame}

\begin{frame}
	\frametitle{$SNR$ calculation}
	\begin{itemize}
		\item Signal-to-noise ratio in decibels:
	\end{itemize}
	\begin{equation}
		{SNR}_{dB} = P_{R, dB} - P_{int, dB}
	\end{equation}
	\begin{description}
		\item[$P_R$] Calculated using link budget model
		\item[$P_{int}$] Noise power, proportional to the receiver noise temperature and the transmission bandwidth
	\end{description}
\end{frame}

\begin{frame}
	\frametitle{$P_E$ calculation}
	\begin{itemize}
		\item Bit error probability for BSFK transmissions:
	\end{itemize}
	\begin{equation}
		P_E = Q\left(\sqrt{\frac{2E_b}{N_0}}\right)
	\end{equation}
	\begin{enumerate}
		\item ratio $E_b / N_0$ derived from $SNR = \rightarrow \frac{E_b}{N_0 B T_b}$
		\item For binary BPSK, $B = 2/T_b$
		\item Factor $B T_b$ is 2, or 3 dB
	\end{enumerate}
\end{frame}

\begin{frame}
	\frametitle{Impact of $d$, $\lambda$, $B$}
	\begin{columns}
		\column{0.5\linewidth}
			\figNoCapt{2.75cm}{d_lambda.png}
			\figNoCapt{2.75cm}{lambda_b.png}
		\column{0.5\linewidth}
			\figNoCapt{2.75cm}{d_b.png}
			\begin{itemize}
				\item $SNR$ and $P_E$ are negatively affected by:
				\begin{itemize}
					\item Higher distance $d$
					\item Lower wavelength $\lambda$
					\item Wider bandwidth $B$ (lower influence)
				\end{itemize}
			\end{itemize}
	\end{columns}
\end{frame}

\begin{frame}
	\frametitle{Impact of $P_T$}
	\begin{itemize}
		\item $P_E$ for $P_T$ at 50 W, 5 W, and 500 mW:
	\end{itemize}
	\begin{equation}
		\code{1.4062e-05   9.2684e-02   3.3768e-01}
	\end{equation}
	\fig{4cm}{pe_over_pt.png}{$P_E$ over values of $P_T$}
\end{frame}

\begin{frame}
	\frametitle{Alternative modulation: ASK}
	\begin{itemize}
		\item Amplitude-shift keying
		\fig{2.5cm}{ask.png}{Bit stream modulated using ASK}
		\item 0-bit represented as 0
		\item 1-bit represented as $A\cos(2 \pi f_c t)$
	\end{itemize}
\end{frame}

\begin{frame}
	\frametitle{Alternative modulation: ASK ($P_E$ calculation)}
	\begin{itemize}
		\item Correlation coefficients:
		\begin{itemize}
			\item $\rho_{12} = \frac{1}{\sqrt{E_1 E_2}} \int_{-\infty}^{\infty} s_1(t) s_2(t) dt = 0$
			\item $R_{12} = \frac{\sqrt{E_1 E_2}}{E_b}\rho_{12} = 0$
		\end{itemize}
		\item $SNR$ to $E_b/N_0$: conversion factor $B T_b = 2$
		\item Bit error probability:
	\end{itemize}
	\begin{equation}
		P_E = Q\left(\sqrt{\left(1 - R_{12}\right) \frac{E_b}{N_0}}\right) = Q\left(\sqrt{\frac{E_b}{N_0}}\right)
	\end{equation}
\end{frame}

\begin{frame}
	\frametitle{Alternative modulation: FSK}
	\begin{itemize}
		\item Frequency-shift keying
		\fig{2.5cm}{fsk.png}{Bit stream modulated using FSK}
		\item 0-bit represented as $A\cos(\omega_c t)$
		\item 1-bit represented as $A\cos((\omega_c + \Delta \omega) t)$
		\item Assumptions: $\omega_c = \frac{2 \pi n}{T}$ and $\Delta\omega = \frac{2 \pi m}{T}$
	\end{itemize}
\end{frame}

\begin{frame}
	\frametitle{Alternative modulation: FSK ($P_E$ calculation)}
	\begin{itemize}
		\item Correlation coefficient $R_{12} = \frac{\sqrt{E_1 E_2}}{E_b}\rho_{12} = 0$
		\item $SNR$ to $E_b/N_0$: conversion factor $B T_b = 2.5$
		\item Bit error probability:
	\end{itemize}
	\begin{equation}
		P_E = Q\left(\sqrt{\left(1 - R_{12}\right) \frac{E_b}{N_0}}\right) = Q\left(\sqrt{\frac{E_b}{N_0}}\right)
	\end{equation}
\end{frame}

\end{document}
