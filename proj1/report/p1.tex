\section{Problem 1.1}

\subsection{\code{rand} and \code{randn}}
The MATLAB functions \code{rand} and \code{randn} are used to generate sequences of random numbers.
\begin{description}
	\item[rand] Uniformly distributed random numbers in the interval $ (0, 1) $
	\item[randn] Normally distributed random numbers with $ \mu = 0 $ and $ \sigma = 1 $
\end{description}

The functions support the same combinations of input arguments.
They can be used to return a single scalar when called with no parentheses, or a matrix of $ n_1 \times n_2 \times \dots \times n_N $ elements.


\subsection{\code{hist}}
The MATLAB function \code{hist} is used to generate the histogram of a sequence of numbers, i.e. the distribution of the values within a series of non-overlapping intervals called bins.
The histogram of a dataset can be used to estimate its probability distribution.

The function supports the following arguments:
\begin{description}
	\item[x] mandatory argument representing the input data
	\item[nbins] number of bins into which the data is divided
	\item[xbins] vector containing the center value of each bin
\end{description}

The following return values can be retrieved:
\begin{description}
	\item[none] if no assignment is specified, the function plots the histrogram
	\item[count] vector containing number of elements in each bin
	\item[centers] vector containing the center value of each bin; can be used, together with count, to manually plot the histogram
\end{description}


\subsection{Examples}
Here there will be shown example usages of the two functions, and proofs of their properties.
The following code snippets generate a vector of 1000 numbers using \code{rand} and \code{randn} respectively, calculates their mean value and plot the histogram of the sequence with 10 bins.

\begin{lstlisting}[language=Octave]
a = rand(1000, 1);
mean(a)
hist(a, 10)
\end{lstlisting}
\fig{5cm}{histrand.png}{Histogram of rand function}

As it can be seen from the figure above, the distribution of the numbers is approximately uniform, with all the values within the interval $ (0, 1) $.
The mean of the sequence is very close to 0.5.

\begin{lstlisting}[language=Octave]
a = randn(1000, 1);
mean(a)
std(a)
hist(a, 10)
\end{lstlisting}
\fig{5cm}{histrandn.png}{Histogram of randn function}

As it can be seen from the figure above, the distribution of the numbers follows a gaussian curve.
The mean of the sequence is very close to 0, and its standard deviation to 1.


\subsection{\code{randn} and normal probability density function}
In the following section, it will be shown how the values returned by \code{randn} follow the probability density function of the normal distribution, which is given by the formula:

\begin{equation}
	\mathcal{N}(\mu, \sigma^2) = \frac{1}{\sqrt{2*\sigma^2*pi}} * \exp(-(x-\mu)^2 / (2*\sigma^2))
\end{equation}

The following code generates a sequence of numbers using \code{randn}, calculates its normalized histogram, and then plots it along with an approximated probability density.
These operations are performed with the standard normal distribution, and with example values of $\mu$ and $\sigma$.

\begin{lstlisting}[language=Octave]
clear;
hold off;
clf;

title({'Histogram and probability density'})
figure(1)

% simulation parameters
avg = [0, 2, -5, .5];
sigma = [1, 5, .5, .1];
sigma2 = sigma.^2;
bins = 1000;

for i = 1:length(avg)
  % generate random sequence of normally-distributed numbers
  a = avg(i) + randn(1e6, 1) * sigma(i);
  % generate its histogram
  [hh, xx] = hist(a, bins);

  % generate probability density function
  x = linspace(min(a), max(a), bins);
  px = 1/sqrt(2*sigma2(i)*pi) * exp(-(x-avg(i)).^2 / (2*sigma2(i)));

  % plot histogram and probability density function
  subplot(1,length(avg),i)
  hold on;
  binwidth = xx(2)-xx(1);
  plot(xx, hh/(sum(hh)*binwidth), ":g")
  plot(x, px, ":b")
  txt = sprintf('{ \\sl N(%.2f, %.2f)}', avg(i), sigma2(i));
  legend({"hist()", txt})
  hold off;
end
\end{lstlisting}
\fig{7cm}{p1a.png}{Examples of histograms of normally distributed values and PDF}

As it can be seen in the figure above, the curves on each plot closely match, indicating that the assumption hold true.


\subsection{Gaussian noise and sinusoidal signals}
This section will analyze the effects of noise contamination using sinusoidal signals as reference.

The following code generates three sinewaves with the same frequency and varying phases.
These waves are to be considered sampled versions of their continuous equivalents.
They are then normalized in order to have a power of 1 mW.
Subsequently, four gaussian noise signals are generated using the \code{randn} function, and scaled to have powers of 1, 10, 100 mW and 1 W respectively.
The noise signal are summed to the original sinewaves, and plotted.

\begin{lstlisting}[language=Octave]
clear;
hold off;
clf;

title({'Gaussian noise and sinusoidal signals'})
figure(1)

% simulation parameters
fs = 44100;
f = 1000;
ph = [0 pi/2 pi];
pwr = [.001 .01 .1 1];

% create a time vector
n = 0:100-1;

% generate and normalize 3 sine signals
sig = zeros(length(n), 3);
for i = 1:3
  sig(:,i) = sin(2*pi*f*n/fs + ph(i));
  sig_norm(:,i) = sig(:,i) / std(sig(:,i)) * sqrt(1e-3);
end

% plot signals with legend
subplot(length(pwr)+1,1,1)
plot(n, sig_norm)
legend({'{sin(\omega*T_s*n)}',
  '{sin(\omega*T_s*n + \pi/2)}',
  '{sin(\omega*T_s*n + \pi)}'})

for i = 1:length(pwr)
  % generate noise signal
  noise = randn(1, length(n))';
  % normalize signal
  noise_norm = noise / std(noise) * sqrt(pwr(i));

  % sum sines and noise
  sig_noise = sig_norm + noise_norm;
  % calculate SNR
  snr(i,:) = var(sig_norm)/var(noise_norm);

  % plot noisy signals
  subplot(length(pwr)+1,1,i+1)
  plot(n, sig_noise)
end

% print SNRs
snr
\end{lstlisting}
\fig{9cm}{p1b.png}{Examples of sinusoidal signals and noise, at various SNR levels}

The figure above shows how difficult it becomes to discern the original sine signal from the noise, let alone identifying its phase.
Moreover, the calculated SNRs match the expected values of 1, 0.1, 0.01, and 0.001.
