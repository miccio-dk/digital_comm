\section{Problem 1.1}

\subsection{\code{rand} and \code{randn}}
The MATLAB functions \code{rand} and \code{randn} are used to generate sequences of random numbers.
\begin{description}
	\item[rand] Uniformly distributed random numbers in the interval $ (0, 1) $
	\item[randn] Normally distributed random numbers with $ \mu = 0 $ and $ \sigma = 1 $
\end{description}

The functions support the same combinations of input arguments.
They can be used to return a single scalar when called with no parentheses, or a matrix of $ n_1 \times n_2 \times \dots \times n_N $ elements.


\subsection{\code{hist}}
The MATLAB function \code{hist} is used to generate the histogram of a sequence of numbers, i.e. the blabla.
It support the following arguments:
\begin{description}
	\item[]
	\item[]
\end{description}


\subsection{Examples}
Here there will be shown example usages of the two functions, and proofs of their properties.
The following code snippets generate a vector of 1000 numbers using \code{rand} and \code{randn} respectively, calculates their mean value and plot the histogram of the sequence.

\begin{lstlisting}[language=Octave]
a = rand(1000, 1);
mean(a)
hist(a)
\end{lstlisting}
\fig{5cm}{histrand.png}{Histogram of rand function}

As it can be seen from the figure above, the distribution of the numbers is approximately uniform, with all the values within the interval $ (0, 1) $.
The mean of the sequence is very close to 0.5.

\begin{lstlisting}[language=Octave]
a = randn(1000, 1);
mean(a)
std(a)
hist(a)
\end{lstlisting}
\fig{5cm}{histrandn.png}{Histogram of randn function}

As it can be seen from the figure above, the distribution of the numbers follows a gaussian curve.
The mean of the sequence is very close to 0, and its standard deviation to 1.


\subsection{\code{randn} and normal probability density function}
In the following section, it will be shown how the values returned by \code{randn} follow the probability density function blabla.

\begin{lstlisting}[language=Octave]
a = randn(1000, 1);
mean(a)
std(a)
hist(a)
\end{lstlisting}
\fig{5cm}{histrandn.png}{Histogram of randn function}
