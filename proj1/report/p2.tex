\section{Problem 1.2}

\subsection{\code{xcorr}}
The MATLAB function \code{xcorr} is used to compute the cross-correlation of discrete-time signals.
Given two discrete-time signals $f_1$ and $f_2$, their cross-correlation is defined as:

\begin{equation}
	r_{f_1, f_2}[k] = \sum_{n=-\infty}^{\infty}f_1[n] \cdot f_2[n+k]
\end{equation}

Where $k$ represents the amount of delay between the two signals.
Such operation is particularly useful to verify whether two signals are related, or to identify specific features of a signal within another.

When such function is applied on the same signal, it takes the name of auto-correlation.
For $k=0$, the auto-correlation value of a signal corresponds to its power.

In MATLAB, the following arguments can be specified:
\begin{description}
	\item[x] if only one input vector is passed, its auto-correlation will be computed
	\item[y] if a second input vector is passed, the cross-correlation between the two will be computed; the vectors must have the same length - if that is not the case, the shortest will be padded with zeros
	\item[maxlag] optionally, a maximum value of $k$ can be specified
\end{description}

The function returns a vector containing the result of the operation over various values of $k$.


\subsection{Autocorrelation of the noisy sinusoidal signals}
In this section, the \code{xcorr} function will be used to reconstruct the sinusoidal components of the signals from the previous problem.

The following code generates sinusoidal signals with various SNR levels, calculates their auto-correlation, and plots them.

\begin{lstlisting}[language=Octave]
clear;
hold off;
clf;
pkg load signal

title({'Autocorrelation of noisy sinusoidal signals'})
figure(1)

% simulation parameters
fs = 44100;
f = 1000;
ph = [0 pi/2 pi];
pwr = [0 .001 .01 .1 1];

% create a time vector
n = 0:100-1;

% generate and normalize 3 sine signals
sig = zeros(length(n), 3);
for i = 1:3
  sig(:,i) = sin(2*pi*f*n/fs + ph(i));
  sig_norm(:,i) = sig(:,i) / std(sig(:,i)) * sqrt(1e-3);
end

for i = 1:length(pwr)
  % generate noise signal
  noise = randn(1, length(n))';
  % normalize signal
  noise_norm = noise / std(noise) * sqrt(pwr(i));

  % sum sines and noise
  sig_noise = sig_norm + noise_norm;

  % generate autocorrelation of noisy signals
  for j = 1:3
    [acor_tmp, lag_tmp] = xcorr(sig_noise(:,j));
    acor(:,j) = acor_tmp;
    lag(:,j) = lag_tmp;
  end

  % plot autocorrelation
  subplot(5,1,i)
  plot(lag, acor)
end
\end{lstlisting}
\fig{9cm}{p2.png}{Autocorrelation of noisy sinusoidal signals, at various SNR levels}

It can be easily noticed how the periodic peaks in the auto-correlation plots become fainter, as the SNR decreases.
In the last two plots, only a major peak at $k=0$ is visible, showing how the signal is mainly composed of noise and therefore uncorrelated with itself.
The distance of the peaks in the first two plots corresponds to the period of the sinusoidal component of the signal.
