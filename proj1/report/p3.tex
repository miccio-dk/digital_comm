\section{Problem 1.3}

\subsection{Detection of pulses through cross-correlation}
In this section, it will be demonstrated how the \code{xcorr} function can be used to identify pulses from pulse-train signals with decreasing SNR levels.

The following code generates a random sequence of nonreturn-to-zero pulses between -1 and 1, of given length.
Gaussian noise is applied to the pulse-train signal, following a given SNR value.
The resulting signal is then compared to a reference pulse between 0 and 1 using the \code{xcorr} function.
A sequence of detected pulses is generated by observing the cross-correlation value at given $k$ intervals, provided by the pulse length.
The aforementioned operations are repeated for a selection of SNR levels, and the outcomes are plotted.
The chosen SNR values are 10, 0.1, 0.05, and 0.01.

The resulting plot shows the initial sequence of pulses, the noisy signal, its cross-correlation with the reference pulse, and the detected sequence of pulses.

\begin{lstlisting}[language=Octave]
clear;
hold off;
clf;
pkg load signal

title({'Detection of pulses through cross-correlation'})
figure(1)

% simulation parameters
pulse_len = 100;
pulse_num = 10;
pulse_dc = 0.5;
snrs = [10 .1 .05 .01];

%generate pulse reference
pulse_ref = zeros(1, pulse_len);
pulse_ref(1:(end*pulse_dc)) = 1;

for i = 1:length(snrs)
  % generate pulse-train signal
  pulses = zeros(1, pulse_num * pulse_len);
  pulses_pos = round(rand(1, pulse_num));
  for j = 1:pulse_num
    pulses((j-1)*pulse_len+1 : j*pulse_len) = pulse_ref * pulses_pos(j);
  end
  pulses(pulses==0) = -1;
  % calculate signal power
  pulses_pwr = var(pulses);

  % generate noise signal
  noise = randn(1, pulse_num * pulse_len);
  % normalize signal
  noise = noise / std(noise) * sqrt(pulses_pwr / snrs(i));

  % sum pulse-train and noise
  sig_noise = pulses + noise;

  % generate autocorrelation of noisy signals
  [rxx, lag] = xcorr(sig_noise, pulse_ref);
  % detect pulses:
  % find cross-correlation values for possible pulses
  rxx_pulses = rxx(find(mod(lag, pulse_len) == 0)(pulse_num:end));
  pulses_res = rxx_pulses > 0;

  % plot pulses position
  subplot(length(snrs),4,(i-1)*4+1)
  stem(pulses_pos)
  % plot encoded noisy signal
  subplot(length(snrs),4,(i-1)*4+2)
  plot(sig_noise)
  % plot cross-correlation
  subplot(length(snrs),4,(i-1)*4+3)
  plot(lag(pulse_num*pulse_len:end), rxx(pulse_num*pulse_len:end))
  % plot detected pulses
  subplot(length(snrs),4,i*4)
  stem(pulses_res)
end
\end{lstlisting}
\fig{7cm}{p3.png}{Detection of pulses through cross-correlation, at various SNR levels}

The figure above shows how pulses can be successfully extracted from a noisy source using cross-correlation techniques.

The accuracy of this method has been experimentally verified using a slightly modified version of the code above.
A series of 25 logarithmically spaced values between 10 and 0.001 (corresponding to a range between -30 and 10 dB) has been chosen as SNR, and the overall number of correctly detected pulses counted over a large number of iterations.

The following plot shows how the percentage of correctly detected pulses is a decreases for small SNR levels, but remains always above 50, which corresponds to pure chance.

\fig{4cm}{p3x.png}{Percentage of successfully detected pulses over a range of SNR between -30 and 10 dB}
