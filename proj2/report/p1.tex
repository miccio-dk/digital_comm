\section{Eye Diagram for a Digital Communication Channel}

\subsection{Eye diagram}
A so-called \emph{eye diagram} is the pattern that originates from overlapping a digital signal over the length of one or more transmitted bits.
For several types of digital modulation the plot would show a series of round shapes (eyes) delimited by two rails, hence the name.
The rails are formed by the low-frequency components derived by the bit sequences 00 and 11, whereas the round shapes are formed by the rising and falling edges of the high-frequency bit sequences 01 and 10.

This tool is commonly used to investigate several performance measures of a transmission channel, including noise level, distortion, inter-symbol interference, and synchronization errors.

An eye diagram can be generated with an oscilloscope by enabling infinite persistence, and setting the trigger to react on a separate clock signal.


\subsection{\code{c5ce2.m}: explanation}
Here follows a thoroughly commented version of the provided \code{c5ce2.m} MATLAB script.
The code below generates and plots the eye diagrams of four band-limited signals composed of random sequences of bits.

\begin{lstinputlisting}[language=Octave]{../scripts/1/c5ce2.m}
\end{lstinputlisting}


\subsection{Channel model}
In order to analyze the performances of a given transmission system, a model of the adopted transmission channel has to be devised.

For the purpose of this simulation, the most interesting characteristic of the channel is its bandwidth, and therefore it suitability for transmitting information at a specific rate.
The limitation in bandwidth is obtained by feeding the signal into a third-order Butterworth low-pass filter.

The channel is therefore characterized by its \emph{normalized bandwidth}, which represents the bandwidth as a percentage of the data bit rate; e.g if the bit rate is 8000Hz and the normalized bandwidth is 0.5, the channel will have a cutoff frequency of 4000Hz.

The MATLAB tools for generating digital filters (\code{butter}, \code{ellip}, \code{cheby1}\dots) accepts the normalized frequency as input argument, which is the cutoff frequency in terms of the \emph{Nyquist frequency} (half of the sampling frequency).
The normalized bandwidth is proportional to the normalized frequency by a factor equivalent to the number of samples per bit.


\subsection{\code{c5ce2.m}: different bandwidths}
The following section will present a modified version of the previously introduced script, which can plot the eye diagrams for the normalized bandwidth 0.15, 0.3, 0.7, 1.2, and 4.

In order to do so, the input vector of \code{bw} has been changed to reflect the given values, together with the subplot format, and x-axis label clause.

\begin{lstinputlisting}[language=Octave]{../scripts/1/c5ce2_augm.m}
\end{lstinputlisting}

In order to properly visualize the resulting plots, the original bandwidth values have been removed.
In order to view all the eight eye diagrams, line 11 can be substituted with the following:

\begin{lstlisting}[language=Octave]
    bw = [0.15 0.3 0.7 1.2 4];
\end{lstlisting}


\subsection{\code{c5ce2.m}: plots}
This section will elaborate on the structure and implications of the eye diagrams generated using the scripts above.

The plots are generated by the original and augmented versions of the script respectively, and show the eye diagram for an antipodal baseband transmission at various channel bandwidths.

\fig{9cm}{eye1.png}{Eye diagrams of signals with filter coefficients 0.4, 0.6, 1, and 2}

\fig{9cm}{eye2.png}{Eye diagrams of signals with filter coefficients 0.15, 0.3, 0.7, 1.2, and 4}

When observing an eye diagram, the two main measures of the system performance are the amplitude jitter $A_j$ and the timing jitter $T_j$.

The former measures the difference in amplitude between the low-frequency and high-frequency components of the signals, sampled where the ye is at its maximum opening.
Low channel bandwidth causes the eye to shrink and amplitude jitter to increase, which in turns increases the bit error probability.

The latter parameter represents the time interval at which the zero-crossing may occur.
It is caused by inter-symbol interference, and a high value could result in synchronization issues.


The eye diagrams from both scripts clearly show how a low bandwidth results in overall more jittery and less intelligible signals.
For the normalized bandwidth value of 0.15 and 0.3, 
