\section{The Q function}

\subsection{Normal probability density function 2.1}
\fig{9cm}{normalgaussiangraph.png}{Normal Gaussian pdf graph with defined intervals}
Normal gaussian distribution is one of the most important concept in Communication system and in statistics. It is used as a powerful tool when investigating random signals in communication systems.  Such as investigating behaviour and application of noise signals.

The other reason is that because it is characterised by the limited variables. Such as mean $\mu$ and $\sigma$. It is easier compute and understand when we apply on communication systems.

All the variables that has been used in graphingpdf.m has been explained below;

\begin{description}
	\item [mu] . This is the mean value ($\mu$) for the normal probability density function.
	\item [sigma] This the sensible standard deviation number. ($\sigma$).
	\item [MAX] 50; Maximum x value that x vector will get 
	\item [MIN] -50; Minimum x value that x vector will get
\end{description}

Also general formula for gaussian pdf is ;
y=$f(x|\mu,\sigma)=(\frac{1}{\sigma \sqrt{2 \pi}}) e^{\frac{-(x-\mu)^2}{2(\mu)^2}}$
As we can see and understand from the variables above and the formula that all variables has been specified and we only need $\sigma$, $\mu$ and range.

\subsection{Q(u) function in relation to the normal probability density function}
As we have explained and investigated probability density density function above . We can easily link $Q(u)$ function by exploiting the properties of cumulative distribution function and Probability density function.
\begin{itemize}
	\item The Gaussian probability density function of unit variance and zero mean is $Z(x)=\frac{1}{\sqrt{2\pi}} \exp(\frac{-x^2}{4t})$
	\item And corresponding cumulative distribution function is $P(x)=\int_{-\infty}^{x} Z(t) dt$
	\item The Gaussian $Q$ function is defined as:
		\begin{itemize}
			\item $ Q(x) = 1-P(x)=\int_{x}^{-\infty} Z(t) dt$
		\end{itemize}
\end{itemize}

As we can see from the above mathematical derivation that $Q(u)$ function is directly related to gaussian probability function and cumulative distribution function.
\subsection{ $Q(u)$ function for argument values of relevance to the detection problems for digital communication receivers}
Script file that has been created for the assignment 2.3 and 2.4 has been added to the Appendix. Important concept has been explained and relevant digital communication input values specified.

\subsection{Complementary error function}
To understand the relationship between complementary error function, we need to investigate the $Q(u)$ function and it's relation to 
We know that $Q(u)$ function is :
\begin{itemize}
	\item $Q(z) = \int_{z}^{\infty} \frac{1}{\sqrt{2 \pi}} \exp(\frac{-y^2} {2}) dy$
\end{itemize}

And we have also learnt that complementary error function(erfc) is:
\begin{itemize}
	\item $erfc(z)= \frac{2}{\sqrt{\pi}} \int_{z}^{\infty} \exp(-x^2) dx$
\end{itemize}

From the limits of the integrals in previously defined $Q(z)$ function and $erfc(z)$ function that we can conclude that $Q$ function is directly related to erfc. Mathmetically by combining $Q$ function and $erfc$  we get the following Q function that directly related to erfc:

\begin{itemize}
	\item $Q(z)= \frac{1}{2} erfc(\frac{z}{\sqrt(2)})$
\end{itemize}



\subsection{Plots}
Plotting for the questions 2.3 and 2.4
\fig{9cm}{question2_32_4.png}{Plot for the $Q(u)$ function relevant to the argument values  for the detection problems for digital communication receivers}
\section{Source Code}
Here we have the properly prepared MATLAB codes for the second part of the second project. It has been used for observations, calculations and comparing with specified commands that given in this project.

The code belows computes and graphs normal( Gaussian) probability density function (pdf) in an appropriate intervals

\begin{lstinputlisting}[language=Octave]{../scripts/2/graphingpdf.m}
\end{lstinputlisting}


For the question 2.3 and 2.4 we have created the following code below. This code will plot the $Q(u)$ function for argument values relevant to the digital communication receivers. As we know from previous chapters that there are two common argument values :
\begin{itemize}
	\item $R_12=0 $ (Orthogonal Signals) 
	\item $R_12=-1$ (Antipodal Signals)
\end{itemize}
We also used $z_db$ for the ratio for $E_b/N_o$. We should also notice that $z_db$ is dimensionless ratio.

After that we have investigated the graph for further investigation to see whether we have achieved a satisfactory results for the assignment 2.3 and 2.4. Mathematical calculations are matching up with MATLAB simulation results.

\begin{lstinputlisting}[language=Octave]{../scripts/2/qfunction.m}
\end{lstinputlisting}