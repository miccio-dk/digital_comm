\section{The Matched Filter Base Band Receiver}



\subsection{Principles of a matched filter receiver for binary data in AWGN (Additive White Gaussian Noise) model}
Firstly, we will use the block diagram below to explain important variables for the matched filter receiver for binary data in AWGN. 

A matched filter�is achieved by�correlating�a known�signal(reference signal) with an unknown signal to detect the presence of the template in the unknown signal. The matched filter is the $"optimal"$ linear filter for maximising the SNR( signal-to-noise ratio) in the presence of additive stochastic noise. In this specific case, our additive noise is white gaussian noise.

The matched filter with AWGN is used in communication systems that sends binary messages from transmitter to receiver through AWGN channel.  It is important to firstly define error analysis for general binary noise signals as we try to optimise signal to noise ratio for the filter.

For general error analysis for binary signals in noise are;
\begin{itemize}
	\item Receive $0$ sends $1$
	\item Receive $1$ sends $0$
\end{itemize}
After we received and processed binary signals. The received signal is filtered and filter output is sampled every T seconds. As we can see from the figure that threshold will decide which signal it will pass with respect to gain $k$.

To understand this concept further from error analysis following important variables has been considered. Those are
\begin{itemize}
	\item Minimize the average probability error
	\item Choose the optimal threshold
		\begin{itemize}
			\item Our optimal threshold formula is $0.5[s_1(T)+s_2(T)]$
		\end{itemize}
\end{itemize}

As we know that $P_E$ is a function of the difference between two signals.Therefore , $P_E$ will decrease with increasing argument values. To achieve that we need to make $h(t)$ such that $P_E$ is a minimum when difference between the signals at the output of the filter is maximum. 

Solution for that would be  $h_0(t)=s_2(T-t)-s_1(T-t)$
As we can see that the optimum filter related to  only the input signals.

Important variables has been derived below to see how important they are for the system and enhance our understanding of the block diagram . We will drive formulas for optimum filter and error probability.
\subsubsection{ Derivation for Error Probability for Matched Filter}

Recall $P_E=Q(d/2)$ and we know that our maximum value of the distance is:
\begin{itemize}
	\item $d_m= \frac{2}{N_o}(E_1+E_2-2 \sqrt[]{E_1 E_2 }\rho_12)$
\end{itemize}

We know the energy formulas from the previous chapters. These energy formulas can be used to define $\rho$ formula. Following steps has been done to represent $\rho_12$ in terms of energy. 
\begin{itemize}
	\item $E_1= \int_{t_o}^{t_0+T} s_1^2(t) dt$
	\item $E_2= \int_{t_o}^{t_o+T} s_2^2(t) dt$
	\item $\rho_12=\frac{1}{\sqrt[]{E_1 E_2}} \int_{-\infty}^{\infty} s_1(t) s_2(t) dt$
\end{itemize}
And finalised version of Probability error will become:
\begin{itemize}
	\item $P_E = Q((\frac{E_1+E_2- 2\sqrt[]{E_1 E_2} \rho_12}{2 N_o})^{1/2})$
\end{itemize}



\fig{9cm}{figure_31.png}{Matched Filter for binary data in AWGN}

\subsection{Explanation of the code in \code{c8ce1a.m}}

Here follows a thoroughly commented version of the provided \code{question32.m} MATLAB script.
The code below generates and calculates gain estimate,delay estimate, estimate of px, estimate of py and snr estimate in ratios by given reference and measurement sine waves.

It can be seen below:
\begin{lstinputlisting}[language=Octave]{../scripts/2/question32.m}
\end{lstinputlisting}

\subsection{Create the Matlab user-defined qfn function for code c9ce1.a}
Code has been constructed and located in the source code section of this report. It can be seen from the source part and it works perfectly with the code \code{ce9e1.m}. Detailed explanation given in the source code section.
\subsection{\code{c9ce1a.m}: Extending  the code such that it can take 8 input arguments by given correlation coefficient inputs }

We enter [-1 -0.75 -0.5 0 0.5 0.75 0.8 .995] matrix when we run script and it plots probability of error for different correlation values. For lower correlation values the probability of error decreases since the signals affect each other less. As the correlation values increase, the signals affect each other, e.g., if one is sent both matched filters high values of outputs and with noise the one that isn't sent may become higher easily. In Fig. 6, the results can be seen as expected.

For the part 3.4 and 3.5 \code{question35.m} MATLAB code has been created. It can be seen that code is able take 8 correlation input. It is accessible from the appendix part of this report. Given 8 correlation-coefficient inputs are $[-1, -0.75, -0.5, 0, 0.5, 0.75, 0.8, 0.995]$. Plot has been given below.
 \fig{9cm}{question351.png}{$P_E$ over $E_b/N_o$ graph for investigating how it will change with introduced correlation coefficients}
 
\subsection{ Implementation of Matched filter with a Correlator that includes Cross-Correlations}

In matched filter, $h(t)=s(T-t)$ and output of this filter can be written as convolution
$$h(t)*y(t)=\int\limits^T_0s(T-\tau)y(t-\tau)d\tau$$
where $y(t)$ is received signal. Since system samples signals at t=T and with change of variables to $\alpha=T-\tau$ we can rewrite above equation as
$$v(T)=\int\limits^T_0s(\alpha)y(\alpha)d\alpha$$
where $v(T)$ is sampled version of convolution. The result is integration of the multiplication of received signal with original signal $s(t)$ and it is simply correlation of signals. Therefore, one can implement matched filter receiver with correlator as shown above. 

	
\subsection{Noise presence on the timing synchronisation in the receiver }
 
Even if we manage to recover the timing, it does not guarantee that the correct operation of data-aided frequency estimation algorithms. The reason is that for is the presence of noise on the timing synchronisation in the receiver. Due to fact that for a frequency offset in order of $1/T$ the signal will be severely distorted when it passed through the matched filter. Severely distorted matched filter will not able to maximise the signal to noise ratio in the presence of additive noise.

\subsubsection{Example}

For example, if our noise signal is delta, sampling the matched filters output at some time $T+\delta$ ( where  represents a receiver timing offset due to introduction of noise) will significantly reduce the effective SNR seen by subsequent receiver blocks. This example that linked to theory above proves that it is important to keeping receiver timing offset close to zero as possible and thus delivers motivation for the inclusion of a timing recovery loop in the receiver.


\subsection{Plots}

\subsection{Source Code}
For the question 3.5 following code has been created to take relevant correlation coefficients

\begin{lstinputlisting}[language=Octave]{../scripts/2/question35.m}
\end{lstinputlisting}

For the question 3.3. We have created a user-defined $q(.)$ function because in the \code{ce9e1.m} , q function has not  defined with any error functions in the code. To overcome this problem and make sure code will work. We have created the following $q(.)$ below:

\begin{lstinputlisting}[language=Octave]{../scripts/2/qfn.m}
\end{lstinputlisting}



