\section{The Matched Filter Base Band Receiver}



\subsection{Explain the principles of a matched filter receiver for binary data in white Gaussian noise, thus using the so called AWGN (Additive White Gaussian Noise) model. The description should include a definition of all the necessary variables in the system and a block diagram 3.1}
Firstly, we will use the block diagram below to explain important variables for the matched filter receiver for binary data in AWGN. 

A matched filter�is achieved by�correlating�a known�signal(reference signal) with an unknown signal to detect the presence of the template in the unknown signal. The matched filter is the $"optimal"$ linear filter for maximising the SNR( signal-to-noise ratio) in the presence of additive stochastic noise. In this specific case, our additive noise is white gaussian noise.

The matched filter with AWGN is used in communication systems that sends binary messages from transmitter to receiver through AWGN channel.  It is important to firstly define error analysis for general binary noise signals as we try to optimise signal to noise ratio for the filter.

For general error analysis for binary signals in noise are;
\begin{itemize}
	\item Receive $0$ sends $1$
	\item Receive $1$ sends $0$
\end{itemize}
After we received and processed binary signals. The received signal is filtered and filter output is sampled every T seconds. As we can see from the figure that threshold will decide which signal it will pass with respect to gain $k$.

To understand this concept further from error analysis following important variables has been considered. Those are
\begin{itemize}
	\item Minimize the average probability error
	\item Choose the optimal threshold
		\begin{itemize}
			\item Our optimal threshold formula is $0.5[s_1(T)+s_2(T)]$
		\end{itemize}
\end{itemize}

As we know that $P_E$ is a function of the difference between two signals.Therefore , $P_E$ will decrease with increasing argument values. To achieve that we need to make $h(t)$ such that $P_E$ is a minimum when difference between the signals at the output of the filter is maximum. 

Solution for that would be  $h_0(t)=s_2(T-t)-s_1(T-t)$
As we can see that the optimum filter related to  only the input signals.

Important variables has been derived below to see how important they are for the system and enhance our understanding of the block diagram . We will drive formulas for optimum filter and error probability.
\subsubsection{ Derivation for Error Probability for Matched Filter}

Recall $P_E=Q(d/2)$ and we know that our maximum value of the distance is:
\begin{itemize}
	\item $d_m= \frac{2}{N_o}(E_1+E_2-2 \sqrt[]{E_1 E_2 }\rho_12)$
\end{itemize}

We know the energy formulas from the previous chapters. These energy formulas can be used to define $\rho$ formula. Following steps has been done to represent $\rho_12$ in terms of energy. 
\begin{itemize}
	\item $E_1= \int_{t_o}^{t_0+T} s_1^2(t) dt$
	\item $E_2= \int_{t_o}^{t_o+T} s_2^2(t) dt$
	\item $\rho_12=\frac{1}{\sqrt[]{E_1 E_2}} \int_{-\infty}^{\infty} s_1(t) s_2(t) dt$
\end{itemize}
And finalised version of Probability error will become:
\begin{itemize}
	\item $P_E = Q((\frac{E_1+E_2- 2\sqrt[]{E_1 E_2} \rho_12}{2 N_o})^{1/2})$
\end{itemize}



\fig{9cm}{figure_31.png}{Matched Filter for binary data in AWGN}

\subsection{Explain the code in \code{c8ce1a.m} in greater detail than the present comments. Insert the comments directly in a renamed version of the script 3.2}

Here follows a thoroughly commented version of the provided \code{question32.m} MATLAB script.
The code below generates and calculates gain estimate,delay estimate, estimate of px, estimate of py and snr estimate in ratios by given reference and measurement sine waves.

It can be seen below:
\begin{lstinputlisting}[language=Octave]{../scripts/2/question32.m}
\end{lstinputlisting}

\subsection{3.3}
\subsection{\code{c9ce1a.m}: Extend the code in c9ce1a.m such that the number of input correlations can be up to 8. Presently 4 is the maximum of input arguments 3.4}

For the part 3.4 \code{question35.m} MATLAB code has been created. It can be seen that code is able take 8 correlation input. It is accessible from the appendix part of this report.
\subsection{Apply the following correlation coefficients in the c9ce1a script  $[-1 -0.75 -0.5 0 0.5 0.75 0. 8 .995]$ and insert the resulting plot in the project report. Explain the plot and verify that it is correct 3.5}
 \fig{9cm}{question351.png}{$P_E$ over $E_b/N_o$ graph for investigating how it will change with introduced correlation coefficients}
%tomorrow explain with providing the book graph example.
\subsection{ Explain how a matched-filter receiver can be implemented by a correlator. The explanation should include the most important formulas leading to the matched filter implementation using a cross correlation Part 3.6}
\fig{9cm}{correlator_matched_filter.png}{Correlator Matched Filter}
As we can from the Figure 3 that we can implement the matched filter by implementing the correlation detectors. The cross-correlator does the cross-correlation between the noisy signal and noiseless signal. 

If we look at the matched filter, it does the convolution between the received signal and the time-reversed copy of the reference signal. We can mathematically prove that both matched filter and the cross-correlator will give us same output signal values. To prove this we will derive mathematical  derivation. 

Suppose we have received $r(t)$ and is passed through parallel bank of $N$ ?crosscorrelators? which basically compute the prediction of $r(t)$ onto the $N$-basis functions as $f_n(t)$.This  can be seen in Figure 3.

For cross-correlator  signal is now represented by $s_m(t)$ with components $s_{mk}(t)$. $s_{mk}(t)$ values depend on how many $M_{signal}$ is transmitted. Now other components of cross-correlator will be explained below for the mathematical derivation for  the output signal.

\begin{itemize}
	\item $\int_{0}^{T} r(t) f_k(t)=\int_{0}^{T} [s_m(t)+n(t)] f_k(t) dt$
	\item $r_k(t)=s_m(t)+n_k(t)$
	\item $\int_{0}^{T} s_m(t) f_k(t) $ for the values of $k= 1,2,3,4,5,...$
	\item $\int_{0}^{T} n(t) f_k(t) $ for the values of $k=1,2,3,4,5.....$
\end{itemize}

Now our signal has been represented by the $s_m(t)$ and the $n(k)$ (random variables as noise)

Now the signal can be represent $r(t)$ as

\begin{itemize}
	\item $r(t)= \sum\limits_{k=1}^N = s_m(t) f_k(t)+ n(t)'+\sum\limits_{k=1}^N = n_k(t) f_k(t)$
	\item $\sum\limits_{k=1}^N = r_k(t) f_k(t)$
\end{itemize}

As we can see that $n(t)'$ is irrelevant to which signal is going to transmitted. The decision will be made upon entirely on correlator output and the basis functions. 

Now we can compare with Matched filter output. 

For matched filter, we use $N$ bank linear filters. Impulse responses of the $N$ filters are;
\begin{itemize}
	\item $h_k(t)=f_k(T-t)$ for invertals between , $0<t<T$
\end{itemize}

The output filters become:
\begin{itemize}
	\item $y_k(t)=\int_{0}^{T} r(\tau) f_k(T-\tau)d(\tau)$
	\item $\int_{0}^{T} r(\tau) f_k(T-t+\tau)d(\tau)$ for the $k= 1,2,3,4,5,...N$
\end{itemize}
If we sample outputs of the linears at $t=T$
\begin{itemize}
\item $y_k(T)=\int_{0}^{T} r(\tau) f_k(\tau)d(\tau)$ which will eventually become $r_k$
\end{itemize}

Hence our theoritical approach has been proved by mathematical derivation of matched filter and cross-correlator. Both have same output even though calculations have been done in a different way for both.
	
\subsection{ Explain what the consequences are with respect to a matched-filter receiver if there is noise on the timing synchronisation in the receiver Part 3.7}
 
Even if we manage to recover the timing, it does not guarantee that the correct operation of data-aided frequency estimation algorithms. The reason is that for is the presence of noise on the timing synchronisation in the receiver. Due to fact that for a frequency offset in order of $1/T$ the signal will be severely distorted when it passed through the matched filter. Severely distorted matched filter will not able to maximise the signal to noise ratio in the presence of additive noise.

\subsubsection{Example for the Part 3.7}

For example, if our noise signal is delta, sampling the matched filter?s output at some time $T+\delta$ ( where  represents a receiver timing offset due to introduction of noise) will significantly reduce the effective SNR seen by subsequent receiver blocks. This example that linked to theory above proves that it is important to keeping receiver timing offset close to zero as possible and thus delivers motivation for the inclusion of a timing recovery loop in the receiver.


\subsection{Plots}

\subsection{Source Code}
For the question 3.5 following code has been created to take relevant correlation coefficients
\begin{lstinputlisting}[language=Octave]{../scripts/2/question35.m}
\end{lstinputlisting}

