\documentclass{beamer}
\usetheme{Berlin}
\usecolortheme{beaver}
\usepackage{graphicx}
\usepackage[export]{adjustbox}
\usepackage{tikz}
\usetikzlibrary{arrows}
\usepackage{amsmath}
\usepackage{lmodern}% http://ctan.org/pkg/lm
\usepackage{mathtools}
\usefonttheme{professionalfonts}

\title{Appendix: noise}
\subtitle{}
\author[Riccardo \and Eren]{Riccardo~Miccini\inst{1} \and Eren~Can~\inst{1}}
\institute[DTU]
{
	\inst{1}
	Technical University of Denmark\\
	Digital Communication
}
\date{\today}
\subject{Digital Communication}

\tikzstyle{int}=[draw, fill=blue!20]
\tikzstyle{every node}=[font=\tiny]

\begin{document}
\frame{\titlepage}

% ch A.2
\begin{frame}
	\frametitle{Characterization of noise in systems}
	\begin{itemize}
		\item
	\end{itemize}
\end{frame}


% ch A.2.1
\begin{frame}
	\frametitle{Noise Figure of a System}
	\begin{itemize}
		\item 
	\end{itemize}
\end{frame}


% ch A.2.3
\begin{frame}
	\frametitle{Noise Temperature}
	\begin{itemize}
		\item
	\end{itemize}
\end{frame}


% ch A.2.3
\begin{frame}
	\frametitle{Free Space Propagation Example}
	\begin{itemize}
		\item As a final work for the noise calculation, we will investigate the "free-space electromagnetic-wave propagation channel.
		\item To understand it fully on a practical example, we will investigate the communication tie between a synchronous-orbit relay satellite and a low-orbit satellite.
		\end{itemize}
\end{frame}




\end{document}
